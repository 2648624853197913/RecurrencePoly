\documentclass[a4paper,12pt]{article}

%% Language and font encodings
\usepackage[english]{babel}
\usepackage[utf8]{inputenc}
%\usepackage[T1]{fontenc}

%% useful includes
\usepackage{amsmath}
\usepackage{amsfonts}
\usepackage{amssymb}
\usepackage{amsthm}
\usepackage{mathtools}

\usepackage[authoryear]{natbib}
\bibliographystyle{abbrv}

\usepackage{graphicx}
\usepackage{enumitem}
\usepackage{hyperref}


% %% Sets page size and margins
%\usepackage[a4paper,top=3cm,bottom=2cm,left=3cm,right=3cm,marginparwidth=1.75cm]{geometry}
\setlength{\parskip}{0.2em}




%% some useful commands

\renewcommand{\P}{\mathbf{P}}
\newcommand{\NP}{\mathbf{NP}}
\newcommand{\UP}{\mathbf{UP}}
\newcommand{\NPL}{\mathbf{NPL}}
\newcommand{\PLT}{\mathbf{PLT}}
\newcommand{\NPLMV}{\mathbf{NPLMV}}
\newcommand{\p}{\text{promise-}}
\newcommand{\co}{\text{co-}}
\renewcommand{\O}{\mathcal{O}}
\newcommand{\Lf}{\text{Leaf}}
\newcommand{\BL}{\text{BLeaf}}
\newcommand{\RLf}{\text{RLeaf}}
\newcommand{\RBL}{\text{RBLeaf}}
\newcommand{\ls}{\text{leafstring}}
\newcommand{\polylog}{\text{polylog}}
\newcommand{\faP}{\forall^{\text{P}}\cdot}
\newcommand{\exP}{\exists^{\text{P}}\cdot}
\newcommand{\faPL}{\forall^{\text{PL}}\cdot}
\newcommand{\exPL}{\exists^{\text{PL}}\cdot}
\newcommand{\tr}{\mathrm{tr}}
\newcommand{\trace}{\mathrm{tr}}

\newcommand{\In}{I n p u t: }
\newcommand{\Out}{O u t p u t: }
\newcommand{\Alg}{A l g o r i t h m: }



%% environments

\newtheorem{thm}{Theorem}[section]
\newtheorem{claim}[thm]{Claim}
\newtheorem{conj}[thm]{Conjecture}
\newtheorem{cor}[thm]{Corollary}
\newtheorem{definition}[thm]{Definition}
\newtheorem{lemma}[thm]{Lemma}
\newtheorem{problem}[thm]{Problem}

\theoremstyle{remark}
\newtheorem{remark}[thm]{Remark}
\newtheorem{example}[thm]{Example}
\newtheorem{examples}[thm]{Examples}

%\pagestyle{fancy}
%\fancyhead{}
%\fancyfoot{}
%\fancyhead[R]{\thepage}



\begin{document}

\section*{Collection of Equations}

Let $G = (V,E)$ be a graph on $n \coloneqq |V|$ nodes. Denote its complement by $G^c$ (i.e. $G^c$ has the same nodes as $G$ and two nodes are adjacent in $G^c$ if and only if they are not adjacent in $G$). Recall that
$$
r^G_{uv}(x) \coloneqq \chi_G(x) \cdot x^{-1} \cdot H_{uv}(x),
$$
where $H_{uv}(x)$ is the generating function of the sequence of walks from node $u \in V$ to node $v \in V$ (actually, the map $\sigma(x) \mapsto \chi_G(x) \cdot x^{-1} \cdot \sigma(x)$ defines an isomorphism from the set of linear recurrence sequences having characteristic polynomial $\chi_G(x)$ into the quotient ring $\mathbb{Z}[x]/_{\chi_G(x)}$). If the graph in question is unambiguous, we will suppress it in the notation of $r^G_{uv}(x)$ and simply write $r_{uv}(x)$ instead.

Here is a collection of facts/equations that arise in this context. Proofs can be provided if necessary.
\begin{enumerate}

\item 
This definition extends by linearity to the more general setting of walks from a subset $S\subseteq V$ to a subset $T\subseteq V$ (in particular, it holds that $r_{S,T}(x) = \sum_{u\in S}\sum_{v\in T}r_{uv}(x)$). Furthermore, define $r_G(x) \coloneqq r_{V,V}(x) = \sum_{(u,v)\in V^2}r_{uv}(x)$.

\item 
The degree of the polynomial $r_{uv}(x)$ is given by $\deg(r_{uv}) = n - 1 - \textrm{dist}(u,v)$.

\item
The recurrence polynomial (i.e. least characteristic polynomial) of the sequence of walks from $S$ to $T$ equals
$$
\varrho_{S,T}(x) = \frac{\chi_G(x)}{\gcd (\chi_G(x),r_{S,T}(x))}.
$$
Therefore, $\varrho$ can be seen as some kind of waste product of $\chi$ and $r$, since they seem to be better controllable.

\item 
The leading coefficient of $r_{uv}(x)$ equals the number of shortest paths from $u$ to $v$ (unfortunately this does not generalize to the lower coefficients).

\item 
Let $P_{uv}$ denote the number of paths from $u$ to $v$ (i.e. walks visiting each node only once). Then it holds that
$$
r_{uv}(x) = \sum_{p \in P_{uv}} \chi_{G \backslash p} (x)
$$
for all $u,v \in V$.

\item 
In particular, this implies
$$
r_{vv}(x) = \chi_{G \backslash\{v\}}(x)
$$
for all $v \in V$.

\item \label{chiGc}
The global polynomial $r_G(x)$ can be computed as
$$
r_G(x) = -\chi_G(x) + (-1)^n \chi_{G^c}(-1-x).
$$
Therefore, it furthermore holds that
$$
r_G(x) = (-1)^{n-1} r_{G^c}(-1-x).
$$

\item
Consequently, we can compute the derivatives of $\chi_G(x)$ and $r_G(x)$ as follows:
\begin{align*}
\chi_G'(x) &= \sum_{v\in V}r_{vv}(x) = \sum_{v\in V} \chi_{G \backslash\{v\}}(x),\\
r_G'(x) &= \sum_{v\in V} r_{G \backslash\{v\}}(x).
\end{align*}

\item 
Let $G_1$ and $G_2$ be undirected graphs and denote their join by $G_1 \triangledown G_2$ (i.e. $G_1 \triangledown G_2 = \left(G_1^c \cup G_2^c\right)^c$). Then the following holds:
\begin{align*}
\chi_{G_1\cup G_2}(x) &= \chi_{G_1}(x) \cdot \chi_{G_2}(x),\\
r_{G_1\cup G_2}(x) &= r_{G_1}(x)\cdot\chi_{G_2}(x) + \chi_{G_1}(x)\cdot r_{G_2}(x),\\
\chi_{G_1 \triangledown G_2}(x) &= \chi_{G_1}(x) \cdot \chi_{G_2}(x) - r_{G_1}(x)\cdot r_{G_2}(x),\\
r_{G_1 \triangledown G_2}(x) &= -\chi_{G_1}(x)\cdot \chi_{G_2}(x)+\chi_{G_1}(x)\cdot r_{G_2}(x) + r_{G_1}(x)\cdot\chi_{G_2}(x) + 2\cdot r_{G_1}(x) \cdot r_{G_2}(x)
\end{align*}
In particular, from this we can derive that the spectrum of $G_1 \triangledown G_2$ contains all non-main eigenvalues of $G_1$ and $G_2$, but none of the main eigenvalues of $G_1$ or $G_2$. (careful, here the same eigenvalue with multiplicity $k$ is counted as $k$ distinct eigenvalues). Furthermore, if $\lambda$ is a non-main eigenvalue of some $G_i$, it is also a non-main eigenvalue of $G_1 \triangledown G_2$. Nevertheless, there can exist non-main eigenvalues of $G_1 \triangledown G_2$ which do not come from $G_1$ or $G_2$.

This simple example already shows that the polynomials $r_{S,T}(x)$ can get messy really fast, when the structure of the graph is slightly modified.

\item 
For a subset $S \subseteq V$, let $G_S$ be the graph consisting of $G$ with an additional node connected to the nodes in $S$. Then it holds that
\begin{align*}
\chi_{G_S}(x) &= x\cdot \chi_G(x) - r_{SS}(x),\\
r_{G_S}(x) &= (x+1)\cdot r_G(x) + \chi_G(x) + r_{SS}(x) + (-1)^nr_{\bar{S}\bar{S}}^{G^c}(-1-x).
\end{align*}

If $S = \{v\}$ is a single node, then we can rewrite this as
\begin{align*}
\chi_{G_v}(x) &= x \cdot\chi_G(x)-\chi_{G\backslash\{v\}}(x),\\
r_{G_v}(x) &= x \cdot r_G(x) + \chi_G(x) - r_{G\backslash\{v\}}(x)+2r_{v,G}(x).
\end{align*}

If $S = N(v)$ for some node $v\in V$, this simplifies to the equations
\begin{align*}
x^2\cdot\chi_{G\backslash\{v\}}(x) - 2x\cdot\chi_G(x) + \chi_{G_S}(x) &= 0,\\
x^2\cdot r_{G\backslash\{v\}}(x) - 2x\cdot r_G(x) + r_{G_S}(x) &= 0.
\end{align*}

\item 
Define the matrix $R_G(x) \in \mathbb{Z}[x]^{n\times n}$ to be the collection of all polynomials $r_{uv}(x)$, i.e.
$$
\left(R_G(x)\right)_{u,v} \coloneqq r_{uv}(x).
$$
For an arbitrary polynomial $p(x) = \sum_{i=0}^d p_ix^i \in \mathbb{Z}[x]$, define its truncation on the $j$-th level as
$$
\tr_jp(x) = \sum_{i=0}^{d-1-j}p_{i+j+1}x^i.
$$
In particular, $\tr_{-1}p(x) = p(x)$ and $tr_{d-1}p(x) = 1$. Then it holds that
$$
R_G(x) = \sum_{i=0}^{n-1}\tr_i\chi_G(x)A^i = \sum_{i=0}^{n-1}\tr_i\chi_G(A)x^i.
$$

\item 
Let $\lambda_1, \ldots, \lambda_n$ be the eigenvalues of $G$, then the characteristic polynomial $\chi_{R_G(x)}(y) \in \mathbb{Z}[x,y]$ can be computed as
$$
\chi_{R_G(x)}(y) = \prod_{i=1}^n\left(y - \frac{\chi_G(x)}{x-\lambda_i}\right) = \frac{y^n \chi_G\left(x-\frac {\chi_G(x)}y\right)}{\chi_G(x)}.
$$
In particular, this shows that the eigenvalues of $R_G(x)$ are exactly the polynomials $\frac{\chi_G(x)}{x-\lambda}$ for the eigenvalues $\lambda$ of $G$. Furthermore, it holds that $\det\left(R_G(x)\right) = \chi_G(x)^{n-1}, \trace\left(R_G(x)\right) = \chi_G'(x)$ and $\trace\left(R_G(x)^2\right) = \chi_G'(x)^2 - \chi_G''(x)\chi_G(x)$ (local version of the last equation: $\sum_{(u,v)\in V^2}r_{uv}(x)^2 = \chi_G'(x)^2 - \chi_G''(x)\chi_G(x)$).

One can even show that the eigenvectors of $G$ with respect to the eigenvalue $\lambda$ coincide with the eigenvectors of $R_G(x)$ with respect to the eigenvalue $\frac{\chi_G(x)}{x-\lambda}$.

\item 
This induces another (rather impractical) way to compute $R_G(x)$: If $P$ diagonalizes the adjacency matrix $A$ (i.e. $P ^\top A P = D$), then
$$
R_G(x) = P \cdot \mathrm{diag}\left(\frac{\chi_G(x)}{x-\lambda_1}, \ldots, \frac{\chi_G(x)}{x-\lambda_n}\right) \cdot P^\top.
$$

\item 
The matrix $R_G(x)$ admits some kind of linearity: For an arbitrary polynomial $p(x) \in \mathbb{Z}[x]$ of degree $d$, it holds that
$$
p(A) \cdot R_G(x) = R_G(x) \cdot p(A) = p(x) R_G(x) - \chi_G(x) \cdot \sum_{i=0}^{d-1} \tr_ip(x)A^i.
$$
Note that this implies linearity over the quotient ring $\mathbb{Z}[x]/_{\chi_G(x)}$.
In particular, one consequence of this is an easier local version. For $u,v \in V$ arbitrary, we have the following equality:
$$
r_{N(u),v}(x) = r_{u, N(v)}(x) = \begin{cases}
x \cdot r_{uu}(x) - \chi_G(x), & u = v\\
x \cdot r_{uv}(x), & u \neq v
\end{cases}.
$$


\item 
There exists a direct way to compute $r_{uv}(x)$ for particular $u \neq v \in V$. In fact, it holds that
$$
r_{uv}(x)^2 = \chi_{G\backslash \{u\}}(x)\cdot \chi_{G\backslash\{v\}}(x) - \chi_{G\backslash\{u,v\}}(x)\cdot\chi_G(x).
$$

\item 
This can be carried out to the polynomial $r_G(x)$ instead of $\chi_G(x)$, i.e. we define a matrix $M_G(x) \in \mathbb{Z}[x]^{n \times n}$ by
$$
\left(M_G(x)\right)_{u,v} \coloneqq \begin{cases}
-r_{G\backslash\{u\}}(x), &u = v\\
\sqrt{r_{G\backslash\{u\}}(x) \cdot r_{G\backslash\{v\}}(x) - r_{G\backslash \{u,v\}}(x) \cdot r_G(x)}, &u \neq v
\end{cases}.
$$
Then (in analogy to point \ref{chiGc}) we can write $M_G(x)$ as
\begin{align*}
M_G(x) &= R_G(x) + (-1)^n \cdot R_{G^c}(-1-x),\\
M_G(x) &= (-1)^n \cdot M_{G^c}(-1-x).
\end{align*}

Note that, apart from the sign change on the main diagonal, the matrix $M_G(x)$ is obtained from $r_G(x)$ in the same way as $R_G(x)$ from $\chi_G(x)$ (maybe this generalizes to some kind of morphisms $\mathcal{G}\to\mathbb{Z}[x]$, where $\mathcal{G}$ denotes the set of all undirected graphs).

Since $(1, \ldots, 1) \cdot M_G(x) \cdot (1, \ldots, 1)^\top = 0$, this construction cannot be carried out further.
\end{enumerate}

\end{document}
